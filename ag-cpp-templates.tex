\documentclass{llncs}

\usepackage{graphicx}
\usepackage{listings}


\lstset{
  language=C++,
  basicstyle=\normalsize\ttfamily
}


\newcommand*{\Cpp}{C\texttt{++}}

\title{Embedded Attribute Grammars as \Cpp{} Templates}

\author{Arie Middelkoop}
\institute{\email{amiddelk@gmail.com}}

\begin{document}

  \maketitle

  \begin{abstract}
  Attribute grammars have been implemented as a statically-typed,
  embedded domain specific language
  in several programming languages that feature meta programming facilities.
  \Cpp{} templates are considered to be an expressive meta programming language.
  In this article, we give an outline of such an embedding using \Cpp{} templates.
  \end{abstract}
 
\section{Introduction}
\label{sec:introduction}

  This article considers what would be Doaitse's next `best' programming language 
  to embed attribute grammars in, when Haskell~\cite{DBLP:conf/sblp/VieraS12} is taken aside.
  Among the list of possibilities, we would normally not consider \Cpp, but
  there is actually some justification for it:
  \begin{itemize}
  \item The embedding focusses on meta programming, which in case of
    \Cpp{} involve templates, at least when we assume that the embedding
    is to be statically typed.
    The template language of \Cpp{} is purely functional. This is a must-have for
    somebody that is such a prominent preacher of functional programming.
  \item The templates were not intended as a programming language, turning
    working on the edges of what is possible in practice into a small
    adventure.
  \item
    Their design and implementation in compilers is so poor that somebody
    should complain to the standardisation committees and to compiler
    implementors to accept improvements.
  \item
    When writing a template program, one quickly runs into
    obscure compiler issues and platform-specific quirks, which is a
    reason to stay into contact with former colleagues and students.
  \end{itemize}
  There exists more serious motivation for embedding
  attribute grammars in \Cpp. There are many compilers
  and transformation tools implemented in \Cpp, including the
  large gcc compiler. A lightweight \Cpp{}
  embedding of attribute grammars can for example be useful in the
  description of transformation steps of the latter.

  It is known for a long time that templates are
  an expressive meta language, so it is not surprising that an
  embedding is possible. To do so, we need to:
  \begin{itemize}
  \item Provide an API for the type-level description of an
    attribute grammar (as a type). In particular, we want to
    compose such a type out of individual fragments, such as
    nonterminal, production and attribute declarations (represented
    as types).
  \item Derive data types (e.g. classes in \Cpp) for representing
    the tree as a function of the composed grammar.
  \item Derive the code for decorating the tree as a function of
    the composed grammar. Although we can only derive types using
    templates, this is not a limitation because those types can
    be classes containing functions. Those functions may call
    functions of other derived classes, thereby allowing the
    composition of blocks of code into arbitrary complex
    algorithms.
  \end{itemize}
  As a starting point for doaitse, we sketch the first step in this article,
  and as dependency give an overview of template programming in the
  next section.

\section{Preliminaries: Template Programming}
\label{sect:templates}

  The template language is purely functional~\cite{DBLP:dblp_journals/entcs/Sinkovics11},
  evaluated using a call-by-need strategy with maximal sharing, and untyped.
  The latter makes template programming a rather painful process,
  and proposals that adress this issue (e.g., Concepts) have not been adopted yet.
  Otherwise, template programming is not unlike programming in any other
  functional language, albeit with a clumsy syntax, as we show
  in this section.

  A precondition for programming with templates is a basic understanding of \Cpp{} type
  declarations.
  In particular, the use of typedefs plays a
  major role, as they encodes the concept of let-bindings in templates.
  A confusing point is the frequent need to prefix type expressions that use
  the scope resolution operator \lstinline$::$ with the keyword
  \lstinline$typename$ to explicitly state that the identifier that
  is looked up is a type and not a value or function. As expected, the
  former is usually the case when writing type level computations.
  It is also worth noting that, unlike in C, structs are just
  classes with a default public visibility. In this article, object
  orientation and information hiding do not play an important role, hence
  we will mostly use structs for conciser code.

  Programming with templates allows one to compute a type as a function
  of other types, and templates are the means to express those functions.
  The result of such a `function' is restricted to class
  types (i.e. a struct). This restriction is not an obstacle as we see
  later. We show some templates by example.

  The simplest form is a zero-place function, a constant, which
  is thus actually a conventional type that does not involve
  templates yet:
  \begin{lstlisting}
  struct unit;   // optional forward declaration
  struct unit {  // actual declaration
    // class declaration: body of the template
  };
  \end{lstlisting}
  The class declaration can contain fields and methods, but also
  nested templates, and typedefs. In practice, this means that
  classes encode the concept of (dependend) records, and the
  scope resolution operator \lstinline$::$ as the mechanism
  to dereference or lookup the fields.

  As a more complex example, the identity function is expressed as:
  \begin{lstlisting}
  template<typename x>  // parameter list
  struct identity;      // forward declaration

  template<typename x>  // parameter list
  struct identity {     // actual declaration
    typedef x type;     // field (type)
  };
  \end{lstlisting}
  The template introduces a parameter \lstinline$x$, which can
  be used in type expressions. The typedef binds the type
  expression \lstinline$x$ to the name \lstinline$type$.
  The template can be called in type expressions:
  \begin{lstlisting}
  identity<int>         // the struct itself
  identity<int>::type   // the nested type
  \end{lstlisting}
  Thus, we can express other results than structs (e.g. primitive
  types such as \lstinline$int$) by exposing them as fields.
  We follow the usual convention to store the actual result of the
  function as the field name \lstinline$type$.

  A typical template has the following form:
  \begin{lstlisting}
  template<typename arg1, typename arg2>
  struct my_fun {
    template< /* ... */ >
    struct nested_fun { /* ... */ };

    typedef /* ... type expr ... */ local1;
    typedef /* ... type expr ... */ local2;

    typedef /* ... type expr ... */ type;
  }
  \end{lstlisting}
  As usual in \Cpp, the scope goes from top to bottom,
  and forward declarations can be used to introduce recursion.
  A typical function application has the form of:
  \begin{lstlisting}
  my_fun< /* type expr */, /* type expr */ >::type
  \end{lstlisting}
  When used inside a typedef, we need to make explicit that the
  name \lstinline$type$ represents the name of a type by prefixing
  it with the keyword \lstinline$typename$:
  \begin{lstlisting}
  typedef typename my_fun<int, char>::type result;
  \end{lstlisting}
  In general, templates take one or more types as arguments; in
  the example we showed two.

  The syntax of the template language takes a while to getting used to,
  but in the end it is just functional programming in disguise.
  Note that currying is not supported this way, meaning that all
  function applications need to be fully saturated. The Boost MPL
  library~\cite{boost} provides a slightly more involved encoding of functions
  so that currying can be expressed, but we do not need this
  feature in this article.

  Pattern matching on function parameters can be encoded using
  template specialisation. Noting that the previous type
  \lstinline$unit$ represents a type level zero, and
  \lstinline$identity$ a type level successor, we can implement
  addition of type level natural numbers as follows:
  \begin{lstlisting}
  template<typename x, typename y>
  struct add;  // forward declaration

  template<typename x>
  struct add<x, unit> {   // y=unit
    typedef x type;
  };

  template<typename x, typename z>
  struct add<x, identity<z> > {  // y=identity<z>
    typedef typename add<identity<x>, z>::type type;
  };
  \end{lstlisting}
  The template body with the most specific instantiation is
  chosen, when the choice is unambiguous. Failure to provide such a case
  usually results in a cryptic error message.

  An alternative implementation would wrap the
  identity around the result instead of around the
  first argument. The definition as given above is
  advantageous as it can be written more concisely
  using inheritance:
  \begin{lstlisting}
  template<typename x, typename y>
  struct add<x, identity<y> > : add<identity<x>, y> {
  };
  \end{lstlisting}
  This particular construction often shows up in other
  template programs.

  The Boost MPL library~\cite{boost} provides some
  common abstractions
  to make programming with templates more convenient.  
  It provides an if-then-else function, and
  type level maps with the following API:
  \begin{lstlisting}
  map<>                              // empty map
  insert<mp,pair<key,value>>::type   // insert on map
  has_key<mp,key>::type              // member testing
  fold<mp,state,binfun>::type        // fold function
  \end{lstlisting}
  To implement the DSL, we write functions that fold
  over the structure of the grammar representation,
  which consists largely of type level maps.
  It would be possible to express those folds with
  an attribute grammar, but it should be clear by
  now that we do not intent to cover meta-meta
  attribute grammars in this article.

\section{Embedding}
\label{sect:api}

  The API of the embedding consists of functions to
  construct an attribute grammar description,
  functions to access the attributes from rules of
  the grammar, conventions for encoding those rules,
  and functions to construct the tree and to obtain
  it synthesized attributes.
  We discuss each of these parts in this section.

  \paragraph{Grammar construction.}
  There are several ways to construct descriptions of
  attribute grammars,
  such as oriented per production or per aspect. We do not
  adopt a particular approach, and rather provide a
  low-level API onto which a higher-level API can be build.

  Each of the functions in Figure~\ref{fig:api}
  incorporate an additional asset (nonterminal,
  production, attribute, etc.) to a grammar description,
  and yield the extended
  grammar description. The starting point is an empty
  grammar, available as a constant (i.e. a type).
  The functions are all commutative, i.e. may appear
  in any order. 

  \begin{figure}[ht]
  \begin{lstlisting}
  template<typename name, typename grammar>
  struct decl_nont;

  template <typename nont_name, typename attr_name,
    typename attr_type, typename grammar>
  struct decl_inh;

  template <typename nont_name, typename attr_name,
    typename attr_type, typename grammar>
  struct decl_syn;

  template<typename nont_name, typename prod_name,
    typename term_type, typename grammar>
  struct decl_prod;

  template<typename nont_name, typename prod_name,
    typename child_name, typename child_nont_name,
    typename grammar>
  struct decl_child;

  template<typename nont_name, typename prod_name,
    typename attr_name,
    typename rule_def, typename grammar>
  struct define_syn;

  template<typename nont_name, typename prod_name,
    typename child_name, typename attr_name,
    typename rule_def, typename grammar>
  struct define_inh;
  \end{lstlisting}
  \caption{API for composing attribute grammar descriptions.}
  \label{fig:api}
  \end{figure}

  The difference between \lstinline$decl_syn$ and \lstinline$define_syn$
  is that the former specifies the synthesized attributes of a nonterminal,
  whereas the latter gives their respective definitions per production of the
  nonterminal. The same distinction holds for \lstinline$decl_inh$ and
  \lstinline$define_inh$, except that these are focussed around
  children, i.e. occurrences of nonterminals in the right-hand side of a
  production.

  \begin{figure}[tb]
  \begin{lstlisting}
  struct n_root {}; struct p_root {};
  struct c_root {};

  struct n_tree {};
  struct p_leaf {}; struct p_bin {};
  struct c_left {}; struct c_right {};

  struct i_depth {}; struct s_count {};
  \end{lstlisting}
  \caption{Names used in the example.}
  \label{fig:idents}
  \end{figure}

  The functions of the API take nonterminal, production,
  child and attribute names as parameter. Names in this
  setting are constants, i.e.
  types introduced for the purpose of being a unique
  identifier. Figure~\ref{fig:idents} shows the names that
  we use in the example later in this section.

  \paragraph{Rule encoding.}
  Another topic is how to represent the body of rules, which
  needs to be given as type \lstinline$rule_def$ to the
  functions \lstinline$define_inh$ and \lstinline$define_syn$.
  We encode the body as a static function wrapped in a type.
  The body needs access to the tree, which we pass as parameter.

  Moreover, the type of the tree is derived from the final
  grammar, not from the intermediate grammar that is available
  in the call to \lstinline$define_inh$ or \lstinline$define_syn$.
  We solve this by additional level of indirection, where the
  wrapped type is parameterized with the final grammar, so that
  a rule definition has the following structure:
  \begin{lstlisting}
  struct my_rule {
    template<typename grammar_final>
    struct rule {
      typedef typename node_type<grammar_final,
        nont_name, prod_name>::type node_type;

      static attr_type evaluate(node_type *node) {
        return /* ... */;
      };
    };
  };

  define_syn(n_name, p_name, s_name, my_rule, some_g);
  \end{lstlisting}
  The names \lstinline$rule$ and \lstinline$evaluate$
  cannot be chosen freely, so that when deriving the
  evaluation code from the final grammar
  we can use a statement as follows to call the
  body of the rule:
  \begin{lstlisting}
  val = rule_def::rule<final_grammar>::evaluate(node);
  \end{lstlisting}

  The body of the evaluate function can access the
  values of attributes via the \lstinline$node$
  parameter, using the following API functions
  in the body of rules:
  \begin{lstlisting}
  inh_value<node_type, attr_name>::type::get(node)
  syn_value<node_type, child_name,
    attr_name>::type::get(node)
  \end{lstlisting}
  Note that \lstinline$get$ is a (static) function,
  not a type.

  \begin{figure}[ht]
  \begin{lstlisting}
  typedef empty_grammar g0;
  typedef decl_nont<n_root, g0>::type g1;
  typedef decl_nont<n_tree, g1>::type g2;
  typedef decl_syn<n_tree, s_count, int,
    g2>::type g3;
  typedef decl_inh<n_tree, i_depth, int,
    g3>::type g4;
  typedef decl_prod<n_root, p_root,
    unit, g4>::type g5;
  typedef decl_prod<n_tree, p_leaf,
    unit, g5>::type g6;
  typedef decl_prod<n_tree, p_bin,
    unit, g6>::type g7;
  typedef decl_child<n_root, p_root, c_root,
    n_tree, g7>::type g8;
  typedef decl_child<n_tree, p_bin, c_left,
    n_tree, g8>::type g9;
  typedef decl_child<n_tree, p_bin, c_right,
    n_tree, g9>::type g10;
  typedef define_syn<n_tree, p_leaf, s_count,
    r_zero, g10>::type g11;
  typedef define_syn<n_tree, p_bin, s_count,
    r_sum, g11>::type g12;
  typedef define_inh<n_root, p_root, c_root,
    i_depth, r_init, g12>::type g13;
  typedef define_inh<n_tree, p_bin, c_left,
    i_depth, r_down, g13>::type g14;
  typedef define_inh<n_tree, p_bin, c_right,
    i_depth, r_down, g14>::type g15;
  typedef g15 g_final;
  \end{lstlisting}
  \caption{Example: grammar description.}
  \label{fig:example:grammar}
  \end{figure}

  \begin{figure}[ht]
  \begin{lstlisting}
  struct r_zero {
    template<typename g>
    struct rule {
      typedef typename node_type<g, n_tree, p_leaf>
        ::type t_node;

      int evaluate(t_node *node) {
        return 0;
      }
    };
  };
  \end{lstlisting}
  \caption{Example: trivial rule.}
  \label{fig:example:rule1}
  \end{figure}
 
  \begin{figure}[ht]
  \begin{lstlisting}
  struct r_sum {
    template<typename g>
    struct rule {
      typedef typename node_type<g, n_tree, p_bin>
        ::type t_node;

      int evaluate(node_type<t_node *node) {
        int d = inh_value<t_node, i_depth>
                  ::type::get(node);
        int l = syn_value<t_node, c_left, s_count>
                  ::type::get(node);
        int r = syn_value<t_node, c_right, s_count>
                  ::type::get(node);

        if (d > 2)
          return l + r;
        else
          return 0;
      }
    };
  };
  \end{lstlisting}
  \caption{Example: complex rule.}
  \label{fig:example:rule2}
  \end{figure}

  \paragraph{Example.}
  Before continuing, we provide an example to demonstrate
  the embedding using the functions as given so far.
  The example is about an attribute grammar for binary
  trees where a synthesized attribute \lstinline$s_count$
  represents the total number of leaf nodes at a depth greater
  than 2 given initial depth as inherited attribute \lstinline$i_depth$.
  The example is separated in two parts: Figure~\ref{fig:example:grammar}
  contains the description of the grammar, and
  Figure~\ref{fig:example:rule1} and Figure~\ref{fig:example:rule2} show some of
  the rules.

  We imposed the following conventions to simplify the API: starting
  nonterminals may only have synthesized attributes, and each production
  has exactly one terminal, whose values thus have to be available in
  each node of the tree. Using this convention, inherited attributes
  of the root can be encoded as terminals of the root node, zero
  terminals using the type \lstinline$unit$, and multiple terminals
  using a structured type. In the example, the starting nonterminal
  \lstinline$n_root$ does indeed not take any inherited attributes,
  and all productions have \lstinline$unit$ as type of the terminal.

  \paragraph{Tree construction.}
  Figure~\ref{fig:example:construct} demonstrates
  some templates for constructing the tree. We
  describe these templates in more detail below.

  The following two templates are used to obtain
  types related to a production and nonterminal
  respectively:
  \begin{lstlisting}
  node_type<grammar, nont_name, prod_name>::type
  tree_type<grammar, nont_name>::type
  \end{lstlisting}
  Each \lstinline$node_type$ is a subclass of
  \lstinline$tree_type$ when given the same
  nonterminal name. To construct a node of the
  tree, we use the \Cpp{} constructor of the
  \lstinline$node_type$, that additionally
  takes the value of the terminal as parameter. 
  We use the \lstinline$tree_type$ when the
  distinction in particular node types is
  irrelevant, which is e.g. the case when
  we attach a subtree to a node. We use the
  template \lstinline$attach$ for
  this purpose:
  \begin{lstlisting}
  attach<grammar, nont_name, prod_name, child_name
    ::type::set(parent, child);
  \end{lstlisting}
  Here, \lstinline$parent$ must be a pointer to
  a node of type:
  \begin{lstlisting}
  node_type<grammar, nont_name, prod_name>::type
  \end{lstlisting}
  and \lstinline$child$ must be a pointer to
  any node that is a subclass of:
  \begin{lstlisting}
  tree_type<grammar, name>::type
  \end{lstlisting}
  where \lstinline$name$ is the nonterminal name
  given to the declarationof \lstinline$child_name$.
  In pratice, these pointers should be shared pointers
  of \Cpp, to facilitate automatic garbage collection.
  
  To obtain the synthesized attributes of the root,
  we use the template: 
  \begin{lstlisting}
  result<grammar, nont_name, syn_name>
    ::type::get(root)
  \end{lstlisting}
  The function extracted from the type of the template
  decorates the tree with those attributes necessary to obtain the
  synthesized attribute \lstinline$syn_name$.
  
  \begin{figure}[ht]
  \begin{lstlisting}
  // allocate the nodes
  node_type<g_final, n_root, p_root>
    ::type root(unit);
  node_type<g_final, n_tree, p_leaf>
    ::type middle(unit);
  node_type<g_final, n_tree, p_bin>
    ::type left, right(unit);

  // construct the tree
  attach<g_final, n_root, p_root, c_root>
    ::type::set(&root, &middle);
  attach<g_final, n_tree, p_bin, c_left>
    ::type::set(&middle, &left);
  attach<g_final, n_tree, p_bin, c_right>
    ::type::set(&middle, &right);

  // obtain the resulting attribute
  int d = result<g_final, n_root, s_depth>
            ::type::get(&root);
  \end{lstlisting}
  \caption{Example: tree construction.}
  \label{fig:example:construct}
  \end{figure}

\section{Implementation}
\label{sec:implementation}

  The remainder of this article is deliberately left empty.

\bibliographystyle{splncs}
\bibliography{ag-cpp-templates}

\end{document}

